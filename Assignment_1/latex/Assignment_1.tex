\documentclass[11pt, oneside]{article}
\usepackage{geometry}
\geometry{letterpaper, margin=1in,top=0.75in}
\usepackage[utf8]{inputenc}
\usepackage[english]{babel}

\usepackage{amsmath}
\usepackage{amssymb}
\usepackage{amsthm}

\title{MCSC 6020G - Numerical Analysis \\
        \Large Assignment 1}
\author{Parikshit Bajpai}
\date{}

\newtheorem*{remark}{To prove}

\begin{document}
\maketitle

\section*{Question 1}
\subsection*{(a) Skew-symmetric matrix}

  A square matrix $A$ is called skew-symmetric if $A^T = -A \;\; \text{i.e.} \;\; a_{ij}=-a{ji}$. A general example of such a matrix is:
  \begin{equation*}
      A=
      \begin{bmatrix}
        0              & \lambda_{11}  & \dots   & \lambda_{1n} \\
        -\lambda_{11}  & 0             & \dots   & \lambda_{2n} \\
        \vdots         & \vdots        & \ddots  & \vdots \\
        -\lambda_{1n}  & -\lambda_{2n} & \dots   & 0
      \end{bmatrix}
  \end{equation*}
  where, $\lambda_{ij} \in \mathbb{R}$.

  A specific example of a skew-symmetric matrix is as follows:
  \begin{equation*}
    A=
    \begin{bmatrix}
      0              & \pi         & \dots   & \sqrt{2} \\
      -\pi           & 0           & \dots   & e \\
      \vdots         & \vdots      & \ddots  & \vdots \\
      -\sqrt{2}      & -e          & \dots   & 0
    \end{bmatrix}
  \end{equation*}

\subsection*{(b) Orthogonality}

\begin{remark}
If $B$ is a skew-symmetric matrix, then $A = (\mathbb{I}+B)(\mathbb{I}-B)^{-1}$ is orthogonal, where $\mathbb{I}$ is an identity matrix.
\end{remark}
\begin{proof}
  For a matrix $A$ to be orthogonal, $A^T A = 1$, i.e., $A^T = A^{-1}$. For a skew-symmetric matrix $B$, we can define $ A = (\mathbb{I}+B)(\mathbb{I}-B)^{-1}$. Then,
  \begin{align*}
    A^T   &= \left(\left(\mathbb{I}+B\right)\left(\mathbb{I}-B\right)^{-1}\right)^T \\
          &= \left(\left(\mathbb{I}-B\right)^{-1}\right)^T\left(\mathbb{I}+B\right)^T     && \left(\because (XY)^T = X^T Y^T\right)\\
          &= \left(\left(\mathbb{I}-B\right)^T\right)^{-1}\left(\mathbb{I}+B\right)^T     && \left(\because (X^{-1})^T = (X^T)^{-1}\right)\\
          &= \left(\mathbb{I}^T-B^T\right)^{-1}\left(\mathbb{I}^T+B^T\right)              && \left(\text{Distributivity}\right) \\
          &= \left(\mathbb{I} + B\right)^{-1}\left(\mathbb{I}-B\right)                             && \left(\because B^T = - B\right)\\
          &= \left(\mathbb{I}-B\right)\left(\mathbb{I}+B\right)^{-1}                                        && \left(\text{Commutativity\footnotemark}\right)\\
          &= A^{-1}                                                 && \qedhere
  \end{align*}
\end{proof}
\footnotetext{$(I+B)$ and $(I-B)$ are simultaneously diagonalisable matrices and, in such cases, matrix multiplication is commutative.}

\section*{Question 2}



\end{document}
